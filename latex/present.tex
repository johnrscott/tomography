\documentclass{beamer}

\title{Comparison of estimators in quantum tomography}
\author{John Scott}
\date{\today}

\begin{document}
\begin{frame}
  \titlepage
\end{frame}


\begin{frame}
  \frametitle{Introduction}
  \begin{itemize}
  \item \textbf{The original motivation:} converting classical signal processing to quantum signal processing (originality: $100\%$)
  \item \textbf{The plan for project B:} time varying quantum tomography (originality: $80\%$)
  \item \textbf{Step one of the plan:} look into estimators for steady state quantum tomography (originality: $20\%$)
  \item \textbf{What I actually did:} look into estimators for quantum state tomography with one qubit (originality: $0\%$)
  \end{itemize}
\end{frame}

\begin{frame}
  \frametitle{What is quantum state tomography}
  \begin{itemize}
  \item State estimation. You need many copies of the state because you need to be able to make repeated measurements and build up statistics.
  \item You can use it for characterizing the results from experiments (levels of entanglement, etc.)
  \item 
  \end{itemize}
\end{frame}

\begin{frame}
  \frametitle{Using expectation values to estimate the density matrix}
  The simplest way to do tomography is to just compute a density matrix from expectation values:
  \begin{equation}
    \rho = \frac{1}{2} (I + aX + bY + cZ) = \frac{1}{2}\begin{bmatrix} 1 + c & a - ib \\ a + ib & 1 -c\end{bmatrix}    
  \end{equation}
  The matrix will come out Hermitian and trace 1, but might not have positive eigenvalues (so it isn't `physical'). This problem gets worse depending on how pure the state is. 
\end{frame}

\begin{frame}
  \frametitle{Quality of the esimate (loss functions): distance measures and fidelity}
  \begin{itemize}
  \item{
      Pete's paper has three: the trace distance, the Hilbert-Schmidt distance, and the infidelity:
      \begin{equation}T(A,B) = \frac{1}{2} Tr|A - B|\end{equation}
      \begin{equation}HS(A,B) = \frac{1}{\sqrt{2}} \left[tr(A - B)^2\right]^{\frac{1}{2}}\end{equation}
      \begin{equation}IF(A,B) = 1 - \left[tr\sqrt{A^{\frac{1}{2}}BA^{\frac{1}{2}}}\right]^2\end{equation}
    }
  \item{There are many others: operator norm, fidelity-derived distance, etc.}
  \end{itemize}
\end{frame}

\begin{frame}
  \frametitle{Quality of the linear estimate}
  Put a fancy graph here
\end{frame}

\begin{frame}
  \frametitle{Extended norm minimisation: producing a physical density matrix}
  \begin{itemize}
  \item If you want physical estimates you can try to obtain them from the linear estimate somehow.
  \item The simplest method is to take the linear estimate and project it onto the space of physical states. That's called extended norm minimisation (cite Pete and co.)
  \item There's some fancy confidence interval for the estimate which I haven't got working yet.
  \end{itemize}
\end{frame}

\begin{frame}
  \frametitle{Extended norm minimisation: producing a physical density matrix}
  \begin{itemize}
  \item First obtain the linear estimate $\rho_1$
  \item Then find the closest physical density matrix $\rho_2$ using $$\rho_2 = \min_{\rho} \lVert \rho - \rho_1\rVert_2$$
  \item This result of this estimator is always physical
  \item There is a confidence interval for the estimated density matrix: $$thing$$ 
  \end{itemize}
\end{frame}

\begin{frame}
  \frametitle{Doing ENM}
  You need to find an optimiser which will minimise the L2 distance between your (fixed) linear estimate and a (varying) physical density matrix. Parametrizing the space of physical density matrices can be done by writing (cite whichever paper it came from)
  \begin{equation}
    T = \begin{bmatrix}w&x+iy\\0&z\end{bmatrix}
  \end{equation}
  \begin{equation}
    \rho = T^\dagger T
  \end{equation}
  Then $\rho$ is positive and hermitian. The trace 1 condition is given by requiring that
  \begin{equation}
    w^2+x^2+y^2+z^2=1
  \end{equation}
\end{frame}

\begin{frame}
  \frametitle{Other methods: maximum likelihood}
  You can also use maximum likelihood to get physical estimates. But I didn't.
\end{frame}



\end{document}